\documentclass[a4paper]{article}
\usepackage[14pt]{extsizes} 
\usepackage[T2A]{fontenc}
\usepackage[utf8]{inputenc}
\usepackage[russian, english]{babel}
\usepackage{setspace,amsmath}
\usepackage[left=30mm, top=20mm, right=20mm, bottom=20mm, nohead, footskip=10mm]{geometry} 

\begin{document}
	\begin{spacing}{1.5}
		\begin{center}
			\hfill \break
			\normalsize{Министерство образования и науки Российской Федерации}\\
			\normalsize{Федеральное государственное автономное образовательное учреждение}\\
			\normalsize{высшего профессионального образования}\\				
			\normalsize{«Московский физико-технический институт\\
					(государственный университет)»}\\
			
			\normalsize{Факультет радиотехники и кибернетики}\\
			
			\normalsize{Кафедра интеллектуальных информационных радиофизических систем}\\
			\hfill \break
			\begin{flushright} %Выравнять текст по левому краю блока% 
				Работа допущена к защите\\
				зав. кафедрой\\
				\underline{\hspace{7cm}}\\
				«\underline{\hspace{1cm}}»\underline{\hspace{4cm}}2018г.
			\end{flushright}
			\hfill \break
			\large\textbf{Моделирование траектории движения квадрокоптера в условиях непрерывно изменяющихся параметров внешней среды}\\
			\normalsize{Направление: 03.04.01 - «Прикладные физика и математика»}
			
		\end{center}
		\hfill \break
		\hfill \break
		\normalsize{ 
		\begin{tabular}{lccc}
			Выполнил студент гр. 414 & \underline{\hspace{3cm}} &  &Красников Р.М. \\\\
			Научный руководитель & \underline{\hspace{3cm}} & к.физ.-мат.н. & Кочкаров А.А. \\\\
		\end{tabular}
	}\\
	\begin{center} Москва 2018 \end{center}
	\thispagestyle{empty}
	\end{spacing}

	\newpage
	
	\tableofcontents % Вывод содержания
	\newpage
	
	\newpage
	\section{Введение}
\end{document}